% -*- root: ../informe.tex -*-

\section{Introducción}

En el presente trabajo, nos planteamos conocer mejor la infraestructura
subyacente a las redes de computadoras, capturando y analizando el tráfico que
se produce entre los nodos de las mismas. Nos planteamos tener en cuenta tres
redes de diferentes tamaños, características y tecnologías, para así poder
contrastar los resultados obtenidos.

El elemento en común entre todas las redes estudiadas fue la utilización del
protocolo de red \texttt{IP}, lo cual nos permitió tener en cuenta,
especialmente, el comportamiento del protocolo de control \texttt{ARP}
(\textit{Adress Resolution Protocol}), para analizar la naturaleza de las
comunicaciones entre los nodos y extraer conclusiones acerca de la topología
de la red.

Una herramienta clave para el análisis de los datos fue el empleo de conceptos
de Teoría de la Información, como el de entropía, para evaluar la importancia
relativa de los distintos nodos de la red. Se presentan los resultados así
obtenidos para cada una de las redes, como así también un análisis comparativo
y una interpretación general de los mismos.
