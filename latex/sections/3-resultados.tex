% -*- root: ../informe.tex -*-

\section{Resultados}

\newread\tmp

\openin\tmp=../exp/shopping.s.entropy
\read\tmp to \ShoppingSEntropy
\closein\tmp

\openin\tmp=../exp/shopping.s1.entropy
\read\tmp to \ShoppingSOneEntropy
\closein\tmp

\openin\tmp=../exp/starbucks.s.entropy
\read\tmp to \StarbucksSEntropy
\closein\tmp

\openin\tmp=../exp/starbucks.s1.entropy
\read\tmp to \StarbucksSOneEntropy
\closein\tmp

\openin\tmp=../exp/starbucks.s1prime.entropy
\read\tmp to \StarbucksSOnePrimeEntropy
\closein\tmp

\openin\tmp=../exp/wired_lan.s.entropy
\read\tmp to \WiredLanSEntropy
\closein\tmp

\openin\tmp=../exp/wired_lan.s1.entropy
\read\tmp to \WiredLanSOneEntropy
\closein\tmp

% No hubiera hardcodeado esto, pero no encontré una forma sencilla de hacerlo
% programático

\newcommand\WiredLanNodeCount{207}
\newcommand\ShoppingNodeCount{30}
\newcommand\StarbucksNodeCount{14}


\subsection{Escenario 1 (Red cableada)}

Esta captura se realizó en los laboratorios del Departamento de Computación de la FCEyN, a través de una conexión \emph{Ethernet} con el \emph{switch} de un laboratorio.

En esta red se capturaron muchos más paquetes \emph{broadcast} que \emph{unicast}. Se puede ver en la Figura \ref{res:esc1:fig1}
que los paquetes \emph{unicast} aportan más información que los \emph{broadcast}. Esto se explica porque se trata de una red \emph{switcheada}. Por lo tanto, los paquetes unicast que tienen como destino otros \emph{hosts}, no llegan al equipo de captura. Sin embargo, como en la red hay una cantidad elevada de \emph{hosts} (207), se captura una gran cantidad de \emph{broadcasts}. Como el equipo de captura no se usó para generar una cantidad relevante de tráfico, se capturó más tráfico \emph{broadcast} de la red (aprox 70\%), que el tráfico \emph{unicast} proveniente o destinado a nuestro \emph{host} de captura.

% ¿La entropía de la fuente S es máxima? ¿Que sugiere esto acerca de la red? ¿Está relacionado con el overhead impuesto por la red debido a los protocolos de control (i.e.: ARP)?

Como es posible observar en la Figura \ref{res:esc1:fig1}, la entropía de la fuente $\mathcal{S}$ no es máxima.


\begin{figure}[h]
	\figdef[dim]{figures/wired_lan_s_fig}
	\caption{Cant. de información y entropía de S en red cableada.}
    \label{res:esc1:fig1}
\end{figure}

En la figura \ref{res:esc1:fig2} se muestra un grafo de la red de mensajes \texttt{ARP} que subyace en esta red. Es notable que el grafo no es conexo, sino que se encuentran varias componentes conexas. Esto se debe a que si bien cada laboratorio tiene su propia subred, están todas en la misma \texttt{VLAN}, y por ende, en el mismo \emph{broadcast domain} (capa 2). La subred de cada laboratorio es \texttt{10.2.X.0/24}, donde \texttt{X} es el número de laboratorio, y el gateway de cada red es \texttt{10.2.X.254}. Como vemos en el gráfico, cada componente conexa se corresponde con cada laboratorio, donde las distintas PCs se conectan con el \emph{gateway} de esa subred, que es el que está en el centro de cada componente conexa.

Mirando con más detalle el tráfico \texttt{ARP} de la red, se ven algunas cosas inesperadas en la componente conexa que tiene al nodo \texttt{10.2.203.254} en su centro. Nuestra sospecha es que esos nodos interconectados son los equipos conectados por \emph{Wi-Fi}, y que el nodo central sería el \emph{gateway} de esa red.

% ¿Cómo es el tráfico ARP en la red? ¿Se pueden distinguir nodos? ¿Cuántos? ¿Indica algo la cantidad?
% ¿Se les puede adjudicar alguna función específica? ¿Hay evidencia parcial que sugiera que algún nodo
% funciona de forma anómala y/o no esperada?


%¿Existe una correspondencia entre lo que se conoce de la red y los nodos distinguidos detectados por
%la herramienta? ¿Es posible usar el criterio de distinción propuesto como método para descubrir el/los
%Default Gateway/s de la red? ¿Es preciso?

\begin{figure}[H]
    \figdef[dim]{figures/wired_lan_arp_fig}
    \caption{Red de mensajes \texttt{ARP} para captura de red cableada.}
    \label{res:esc1:fig2}
\end{figure}

% Dada la fuente S1, mostrar la cantidad de información de cada símbolo comparando con la entropía de la fuente.


El modelo que se empleó para analizar la fuente $\mathcal{S}_1$ arrojó una gran cantidad de nodos distinguidos: 12. En la Figura \ref{res:esc1:fig3} podemos ver graficados a los 21 nodos que aportan menos información. Como se ve, entre ellos se encuentran los \emph{gateways} (que terminan en \texttt{.254}), pero también se encuentran otros nodos inesperados:

\begin{enumerate}
% \item \texttt{10.2.1.250}:
\item \texttt{10.2.0.249}, \texttt{10.2.7.249} y \texttt{10.2.3.249}: todas las IPs que vimos con \texttt{10.2.X.249} son de la misma \texttt{MAC} \emph{address}, es decir que son del mismo equipo. Por lo que vimos, se trata de uno de los servidores \emph{Active Directory} de la red. Funcionan también como servidores \texttt{DNS} de la red local.
\item \texttt{10.2.0.67}, \texttt{10.2.0.64}, \texttt{10.2.0.65}: creemos que se trata de servidores, pero no tenemos información suficiente para confirmarlo.
\item \texttt{10.2.1.13}, \texttt{10.2.7.5}, \texttt{10.2.201.234}, \texttt{10.2.200.14} y \texttt{10.2.2.10}: no sabemos por qué figuran como nodos destacados, no encontramos indicios claros sobre su función en la red. Puede tratarse de una anomalía.
\end{enumerate}

\begin{figure}[h]
	\figdef[dim]{figures/wired_lan_s1_fig}
	\caption{Cantidad de información por símbolo y entropía de $\mathcal{S}_1$.}
    \label{res:esc1:fig3}
\end{figure}


En este escenario, utilizar este modelo para la detección de \emph{gateways} no parece ser muy preciso, pero sí sirve como un buen indicador. El problema mayor en esta captura es que el modelo halló muchos nodos distinguidos que no parecen cumplir un rol especial dentro de la topología de la red.


\subsection{Escenario 2 (\emph{Shopping})}

La principal motivación para estudiar esta red fue poder observar un medio con
mucha actividad, el hecho de que se tratara de una red inalámbrica permitió además
contrastar las diferencias frente a la primer captura sobre una red cableada.

Primero se analizarán los resultados obtenidos para la fuente $\mathcal{S}$.

\begin{figure}[h]
	\figdef[dim]{figures/shopping_s_fig}
	\caption{Cantidad de información por símbolo y entropía de $\mathcal{S}$.}
    \label{res:esc2:fig1}
\end{figure}


Como es posible observar en la Figura \ref{res:esc2:fig1} la entropía de
$\mathcal{S}$ no es máxima. La misma posee un valor relativamente inferior lo
cual indica que la fuente es bastante predecible. Se puede apreciar cómo la
cantidad de información que aporta $s_{\texttt{UNICAST}}$ es mínima con respecto
a $s_{\texttt{BROADCAST}}$. Esto se debe al bajo número de paquetes con destino
broadcast en la red.

Una posible hipótesis a este resultado es el hecho de que se trate de la red
inalámbrica de un centro comercial. Es razonable asumir que los hosts conectados
a la misma más que comunicarse entre ellos estarán accediendo mediante el
gateway a Internet.

Para ello, los mismos necesitan la dirección física asociada a la dirección
\texttt{IP} del gateway. Aquí es donde entra en juego \texttt{ARP}, donde los
hosts a través de mensajes broadcast \texttt{WHO-HAS} consultan por ella. Cuando
este mensaje llega al default gateway además de responder con su dirección
física a cada host que consultó por él, el dispositivo guarda la asociación
entre la dirección \texttt{IP} que originó el mensaje junto a su dirección física.

De esta manera, en un principio los únicos mensajes broadcast presentes son los
que resultan de los paquetes \texttt{ARP} generados por los hosts con el fin de
tener la dirección física del gateway. Una vez hecho esto, el gateway ya conoce
a los hosts por lo tanto cualquier comunicación entre ambos resulta en un
intercambio de paquetes unicast.

Habiendo dicho esto, asumiendo que la hipótesis es cierta, sería correcto
afirmar que existe una relación entre los protocolos de control como
\texttt{ARP} y la fuente $\mathcal{S}$. Si hubiera habido comunicación entre los
nodos de la red entonces el número de broadcasts habría sido necesariamente
mayor dado que en lugar de sólo consultar por la dirección física del gateway
también se lo habría hecho por el resto de los hosts.

A continuación se analizarán los resultados obtenidos para la fuente
$\mathcal{S}_1$.

\begin{figure}[h]
	\figdef[dim]{figures/shopping_s1_fig}
	\caption{Cantidad de información por símbolo y entropía de $\mathcal{S}_1$.}
    \label{res:esc2:fig2}
\end{figure}

Estudiando el tráfico \texttt{ARP} de la captura de esta red se notaron algunas
características que merecen ser mencionadas. La primera es que corroborando la
hipótesis sugerida en el análisis anterior, los mensajes del tipo
\texttt{WHO-HAS} iban todos dirigidos al gateway de la red. La segunda,
fuertemente relacionada con la primera, es el hecho de que el gateway es el
único generando respuestas \texttt{IS-AT} a los hosts y prácticamente no realiza
ningún pedido \texttt{WHO-HAS}.

Ambos comportamientos pueden ser justificados con la hipótesis sobre el tipo de
comunicaciones que se establecen en una red inalámbrica dentro de un centro
comercial. Además, buscando información sobre los routers inalámbricos
utilizados para este tipo de instalaciones se encontró que el \emph{timeout} de
las tablas \texttt{ARP} puede llegar a valores de hasta 4 horas. Esto tiene
sentido puesto que estos dispositivos tienen la capacidad de almacenamiento
necesaria para no tener que preocuparse por quedarse sin espacio para las
entradas.

El modelo utilizado para la fuente $\mathcal{S}_1$ dio un total de 30 nodos en la
red. En la Figura \ref{res:esc2:fig2} se muestran los primeros 20 ordenados por
la cantidad de información que brindan, donde se puede apreciar como hay un
único dispositivo cuyo nivel de información está por debajo del de la entropía
calculada.

Este nodo distinguido resulta ser el gateway de la red estudiada. Su bajo aporte
de información tiene sentido puesto que con el modelo utilizado por cada mensaje
\texttt{WHO-HAS} que lo tenga como destino u origen aumenta la probabilidad de
ocurrencia de su símbolo. Por todo lo discutido anteriormente esto
necesariamente lo vuelve el símbolo con mayor probabilidad de aparición y como
consecuencia el que menor información aporta.

Por último, en la Figura \ref{res:esc2:fig3} se tiene la visualización del
tráfico \texttt{ARP} donde el tamaño de los nodos es proporcional a la
probabilidad de sus símbolos en la fuente $\mathcal{S}_1$. Aquí queda más que
evidente el rol de gateway del nodo distinguido, donde todo el tráfico del
protocolo es entre los hosts y el mismo.

Con respecto a la posibilidad de utilizar esta fuente como método para encontrar
los default gateways, como en este caso resultó efectivo uno podría verse
tentado a decir que el mismo es efectivo en la tarea. Sin embargo, existen muchos
factores que podrían llegar a alterar los resultados obtenidos generando
conclusiones incongruentes. Por ejemplo, la captura sobre la cual se realizó
todo este estudio fue hecha en un punto donde llegaba la señal de un único
\emph{access point}. Si hubieran habido más de estos, dependiendo dónde se
realizaba físicamente la captura se habrían obtenido más o menos paquetes
dirigidos a los mismos. Por lo tanto podría haber ocurrido que habiendo dos
default gateways por el simple hecho de que no llegaran a capturarse suficientes
paquetes dirigidos a uno de ellos el símbolo representando al mismo tuviera una
probabilidad asociada muy baja generando entonces mucha información en la fuente
$\mathcal{S}_1$ quedando por encima de la entropía calculada.

\begin{figure}[h!]
	\figdef[dim]{figures/shopping_arp_fig}
    \caption{Red de mensajes \texttt{ARP} para captura del shopping.}
    \label{res:esc2:fig3}
\end{figure}

\vfill % used to avoid funny stretching between paragraphs

\subsection{Escenario 3 (\emph{Starbucks})}

Esta captura se realizó en la red pública de \emph{Wi-Fi} de un café del centro
de Buenos Aires. La duración de la captura fue de 12 minutos y 18 segundos.
Durante este lapso, los usuarios de la red no fueron muy numerosos, lo cual se
refleja en la menor cantidad de nodos detectados en comparación con los
escenarios anteriores. No obstante, se encontraron resultados interesantes, ya
que se observaron varias anomalías en los datos que, gracias a la menor
cantidad de hosts, fueron más sencillas de detectar y estudiar; según se pudo
observar, estas se produjeron tanto por comportamientos anómalos de algunos
hosts, como por la interferencia de paquetes procedentes de una red distinta.

\subsubsection{Fuente $\mathcal{S}$}

Como puede observarse en la Figura \ref{res:esc3:s}, los resultados
obtenidos para la fuente $\mathcal{S}$ son similares a los observados para el
escenario del \emph{shopping}. La entropía de la fuente está relativamente
lejos de ser máxima, indicando que se trata de una fuente bastante predecible.
Esto se debe a que los paquetes de tipo \emph{broadcast} son bastante menos
frecuentes que los de tipo \emph{unicast}, aportando una cantidad de
información considerablemente mayor.

\begin{figure}[H]
    \figdef[dim]{figures/starbucks_s_fig}
    \caption{Cantidad de información por símbolo y entropía de $\mathcal{S}$.}
    \label{res:esc3:s}
\end{figure}

Este fenómeno se puede explicar por diversos motivos. Por un lado, como se
trata de una red de \emph{Wi-Fi} pública, es posible capturar los paquetes de tipo
\emph{unicast} dirigidos hacia cualquiera de sus \emph{hosts}; esto se puede
contrastar con lo que sucede en el caso de la red cableada, donde el
\emph{switching} impide capturar los de otros destinos. Además, también debido
al carácter público de la red, los paquetes de tipo
\emph{broadcast} se utilizan en general para protocolos de control (como
\texttt{ARP}), y estos no producen demasiado \emph{overhead}, ya que las
comunicaciones en general se dan exclusivamente entre el \emph{default gateway}
y los demás nodos.

\subsubsection{Fuente $\mathcal{S}_1$}

La figura \ref{res:esc3:graph} representa a través de un grafo las
comunicaciones entre nodos a partir de paquetes \texttt{ARP}. Al igual que
en los escenarios anteriores, el tamaño de cada nodo es proporcional a su
frecuencia relativa. Una primera mirada rápida revela la presencia de tres
componentes conexas independientes, lo que \emph{a priori} podría significar
que se capturaron paquetes procedentes de tres redes distintas. Esto queda
confirmado al observar las direcciones \texttt{IP} de los \emph{hosts}, que
claramente difieren en la parte que identifica a la red.

\begin{figure}[H]
    \figdef[dim]{figures/starbucks_arp_fig}
    \caption{Red de mensajes \texttt{ARP} para captura del \emph{Starbucks}.}
    \label{res:esc3:graph}
\end{figure}

El modelo empleado para la fuente $\mathcal{S}_1$ permitió detectar 14 nodos,
de los cuales 2 fueron indicados como nodos distinguidos, como puede verse
en la Figura \ref{res:esc3:s1}.

\begin{figure}[H]
    \figdef[dim]{figures/starbucks_s1_fig}
    \caption{Cantidad de información por símbolo y entropía de $\mathcal{S}_1$.}
    \label{res:esc3:s1}
\end{figure}

Estos dos nodos distinguidos, los de \texttt{IP}s \texttt{100.73.63.254} y
\texttt{100.73.57.99}, pertenecen ambos a la misma componente conexa de la
red, y muestran un nivel de actividad similar. Sin embargo, como puede verse
en el grafo de la red, mientras que el nodo \texttt{100.73.63.254} ocupa
un lugar central en la topología (se comunica con otros dos nodos), este no
es el caso del nodo \texttt{100.73.57.99}, lo cual parece ser un indicio de
actividad anómala.

Analizando con cuidado los paquetes \texttt{ARP} capturados, puede apreciarse
que el nodo central (que se trata, seguramente, del \emph{gateway}) parece no
estar respondiendo ninguno de los \texttt{ARP} \emph{requests} enviados por
los demás nodos. Esta es, seguramente, la razón por la que el nodo periférico
realiza \emph{requests} de manera insistente a lo largo de toda la captura,
causando que ambos nodos aumenten notoriamente su frecuencia relativa. En
cuanto a los otros dos \emph{hosts} de esta componente conexa de la red, uno
de ellos presenta como única actividad el \emph{request} \texttt{ARP}; el
otro, por su parte, envía el paquete del \emph{request} \texttt{ARP}
directamente a la dirección física del nodo central, lo cual indica que ya la
conocía y seguramente estaba intentando refrescar su \emph{caché}. Este último
nodo, además, envía una gran cantidad de paquetes de capas superiores (a
diferentes direcciones \texttt{IP}) a través de la dirección física del nodo
central; sin embargo, no recibe ninguna respuesta. Esto parece confirmar que
el nodo central se trata, efectivamente, del \emph{gateway}, y que además, al
momento de realizarse la captura, el mismo no estaba funcionando
correctamente.

Una consecuencia de haber capturado paquetes de más de una red, agravada por
el comportamiento anómalo recién descripto, es que los nodos de las demás
componentes conexas fueron ignorados por completo por el método de detección
de nodos distinguidos. Para poder extraer mejores conclusiones, decidimos
aplicarlo sobre una nueva fuente $\mathcal{S}_1'$, resultante de restringir la
fuente $\mathcal{S}_1$ a los nodos de la componente que tiene como centro al
\emph{host} con \texttt{IP} \texttt{10.251.9.1} (que es la que tiene mayor
cantidad de \emph{hosts}, 8 en total), ignorando los demás símbolos. La Figura
\ref{res:esc3:s1prime} muestra los resultados obtenidos.

\begin{figure}[H]
    \figdef[dim]{figures/starbucks_s1prime_fig}
    \caption{Cantidad de información por símbolo y entropía de $\mathcal{S}_1'$.}
    \label{res:esc3:s1prime}
\end{figure}

Como puede verse en el grafo, la componente analizada también tiene topología
estrellada, y es razonable suponer que el nodo central, que aparece como
distinguido, se trata del \emph{gateway} de la red; esto queda corroborado con
un sencillo análisis del tráfico mantenido con los demás nodos. No obstante,
nuevamente aparece un segundo nodo distinguido, en este caso el de dirección
\texttt{IP} \texttt{10.251.9.160}. Al analizar en detalle el comportamiento de
este \emph{host}, puede verse que esto también se debe a un comportamiento
anómalo: en varias ocasiones generó ráfagas de un gran número de
\emph{requests} consecutivas en un corto período de tiempo (menos de un
segundo), aumentando considerablemente su cantidad de apariciones con respecto
a los demás nodos.

Por último, en la red pueden verse otros dos nodos, que forman una componente
independiente. Ambos tienen una única aparición en la captura, cuando el
nodo \texttt{100.73.22.12} hace un \emph{request} \texttt{ARP} por la
\texttt{IP} \texttt{100.73.22.255}, sin obtener respuesta. Así, decidimos
desestimar estos nodos del análisis, por considerar que no aportan información
relevante y que los datos de la captura realizada son insuficientes para
extraer conclusiones sobre ellos.

Teniendo en cuenta el análisis anterior, podemos decir que en este escenario,
varias de las características de la captura jugaron en contra del desempeño
óptimo del criterio utilizado para distinguir nodos. En primer lugar, la
interferencia entre paquetes de dos redes distintas ocasionó resultados
sesgados hacia una de las redes, que presentaba más tráfico que la otra.
Además, dentro de cada una de las dos redes se distinguieron dos nodos, de
los cuales claramente uno solo cumplía el rol de \emph{default gateway} de
la red, mientras que el otro vio significativamente incrementada su
importancia debido a comportamientos anómalos. Es decir, los resultados
arrojados por la metodología empleada no fueron precisos en este caso.

No obstante, es importante poner esto en perspectiva. En primer lugar, se
trata de una red inalámbrica, donde la poca fiabilidad del medio juega en
contra de la calidad de las capturas, que además se realizaron sin conocer la
ubicación física de los \emph{access points}. Es decir que, por ejemplo,
podrían haber quedado paquetes sin detectar de alguna de las redes. Además, de
haberse realizado una captura de mayor duración, probablemente algunos de los
comportamientos anómalos se habrían visto mitigados, obteniendo así resultados
de mejor calidad.
