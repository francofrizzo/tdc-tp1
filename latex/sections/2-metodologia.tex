\section{Metodología}

Para realizar los experimentos, utilizamos dos herramientas: \emph{Wireshark},
una aplicación de código abierto que permite capturar y analizar los paquetes
que transitan por una red, y \emph{Scapy}, una herramienta escrita en
Python para manipular y trabajar con los paquetes capturados.

Se capturó todo el tráfico de tres redes redes diferentes, en los tres
escenarios que se detallan a continuación.


En los tres casos, se filtraron los resultados para considerar solo los
paquetes relativos al protocolo ARP (\emph{Address Resolution Protocol}).
(Esto capaz queda explicado antes).

Los resultados fueron analizados utilizando conceptos de Teoría de la
Información. Para esto, en cada caso, se modelaron a partir del tráfico
capturado las siguientes dos fuentes de información, considerando distintos
aspectos que se deseaba estudiar:

\begin{enumerate}
    \item \textbf{Fuente $\mathcal{S}$}. Consta de dos símbolos, $\lbrace
    s_{\text{BROADCAST}},\ s_{\text{UNICAST}} \rbrace$. Cada paquete ARP
    capturado en la red se considera un símbolo; aquellos con destino
    \emph{broadcast} (dirección MAC \texttt{ff:ff:ff:ff:ff:ff}) corresponden
    al símbolo $s_{\text{BROADCAST}}$, mientras que los demás corresponden
    al símbolo $s_{\text{UNICAST}}$.

    La probabilidad de la aparición de cada símbolo, y por consiguiente
    la entropía de la fuente, se calculó en base a la frecuencia relativa
    de cada uno de ellos.
    \item \textbf{Fuente $\mathcal{S}_1$}. Esta fuente se modeló teniendo en
    cuenta las direcciones IP de los paquetes ARP capturados.
\end{enumerate}