% -*- root: ../informe.tex -*-

\section{Metodología}

Para realizar los experimentos, utilizamos dos herramientas: \emph{Wireshark},
una aplicación de código abierto que permite capturar y analizar los paquetes
que transitan por una red, y \emph{Scapy}, una herramienta escrita en
Python para manipular y trabajar con los paquetes capturados.

Se capturó todo el tráfico de tres redes redes diferentes, en los tres
escenarios que se detallan a continuación.

\begin{enumerate}
    \item \textbf{Red cableada}. Realizamos esta captura en la red Ethernet
    de los laboratorios del Departamento de Computación (FCEyN, UBA). Se trata
    de una red grande y con bastante tráfico.
    \item \textbf{Red pública de un shopping}. Capturamos también el tráfico
    de la red Wi-Fi pública del \emph{shopping} Unicenter.
    \item \textbf{Red pública de un café}. Por último, realizamos una captura
    en la red pública de Wi-Fi del café \emph{Starbucks} de Av. Callao y Perón.
\end{enumerate}

Las capturas fueron analizadas utilizando conceptos de Teoría de la
Información. Para esto, en cada caso, se modelaron a partir del tráfico
capturado las siguientes dos fuentes de información, considerando distintos
aspectos que se deseaba estudiar:

\begin{enumerate}
    \item \textbf{Fuente $\mathcal{S}$}. Consta de dos símbolos, $\lbrace
    s_{\text{BROADCAST}},\ s_{\text{UNICAST}} \rbrace$. Cada paquete
    capturado en la red se considera un símbolo; aquellos con destino
    \emph{broadcast} (dirección MAC \texttt{ff:ff:ff:ff:ff:ff}) corresponden
    al símbolo $s_{\text{BROADCAST}}$, mientras que los demás corresponden
    al símbolo $s_{\text{UNICAST}}$.

    La probabilidad de la aparición de cada símbolo, y por consiguiente
    la entropía de la fuente, se calculó en base a la frecuencia relativa
    de cada uno de ellos.

    \item \textbf{Fuente $\mathcal{S}_1$}. Esta fuente se modeló teniendo en
    cuenta las direcciones \texttt{IP} de los paquetes \texttt{ARP}
    capturados. Cada dirección de \texttt{IP} es un símbolo en esta fuente. De
    todos los paquetes capturados, se tuvieron en cuenta solo aquellos que
    cumplieran las siguientes condiciones:
    \begin{enumerate}
        \item Fueran \emph{requests} \texttt{ARP} (\texttt{WHO-HAS}).
        Decidimos no tener en cuenta las \emph{replies}, dado que de
        considerarlas tendríamos un sesgo muy fuerte para el equipo que está
        haciendo la captura si la red está switcheada o si es WiFi no abierto.
        Esto se debe a que las \emph{replies} \texttt{ARP} son \emph{unicast},
        y solo llegan a los equipos \texttt{target}, a menos que capturemos en
        un lugar ``privilegiado'' de la topología de la red. Además, incluso
        aunque pudiéramos capturar todas las replies \texttt{IS-AT}, no
        obtendríamos información nueva, ya que las \emph{replies} se
        corresponden con las \emph{requests} (salvo las ``gratuitous'' (ver
        siguiente item).
        \item No fueran ``gratuitous'' (es decir, donde \texttt{target} y
        \texttt{source} son la misma \texttt{IP}), ya que no aportan
        información relevante sobre la topología de la red ni reflejan el
        nivel de actividad de los que los envían.
    \end{enumerate}

    Para calcular la probabilidad de la aparición de cada símbolo se usó la
    frecuencia relativa de cada uno. Se consideraron tres opciones para
    calcular esta última:
    \begin{enumerate}
        \item contar solo cuando una \texttt{IP} aparece como \emph{target},
        \item contar solo cuando una \texttt{IP} aparece como \emph{source}, o
        \item tener en cuenta ambos casos.
    \end{enumerate}
    Decidimos usar esta última opción, porque quien inicia una \emph{request}
    ARP puede ser cualquier nodo, sea distinguido o no. Como nuestro objetivo
    es ver cuáles nodos ``hablan'' más veces con otros nodos de la red, y al
    no tener en cuenta las \emph{replies}, mirando solo el \emph{target} o el
    \emph{source} estaríamos perdiendo información relevante.

    Esta fuente se modeló con el objetivo de estudiar la posibilidad de
    detectar, a partir de la misma, \emph{nodos distinguidos}: aquellos que
    tuvieran un rol destacado en la red. El criterio utilizado fue seleccionar
    como nodos distinguidos a aquellos que aportaran una cantidad de
    información menor que la entropía de la fuente; es decir, menor que la
    media. Desde una interpretación intuitiva, estos son los nodos cuya
    aparición es predecible y no causa ninguna ``sorpresa''. Resulta
    razonable, entonces, señalarlos como los nodos dominantes de la red, al
    menos desde el punto de vista del protocolo \texttt{ARP}. Los experimentos
    realizados permiten contrastar los resultados arrojados por este criterio
    con el conocimiento disponible acerca de la topología y estructura de las
    redes estudiadas.

\end{enumerate}
