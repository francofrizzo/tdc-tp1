\section{Metodología}

Para realizar los experimentos, utilizamos dos herramientas: \emph{Wireshark},
una aplicación de código abierto que permite capturar y analizar los paquetes
que transitan por una red, y \emph{Scapy}, una herramienta escrita en
Python para manipular y trabajar con los paquetes capturados.

Se capturó todo el tráfico de tres redes redes diferentes, en los tres
escenarios que se detallan a continuación.

\begin{enumerate}
    \item \textbf{Red cableada}. Realizamos esta captura en la red Ethernet
    de los laboratorios del Departamento de Computación (FCEyN, UBA). Se trata
    de una red grande y con bastante tráfico.
    \item \textbf{Red pública de un shopping}. Capturamos también el tráfico
    de la red Wi-Fi pública del \emph{shopping} Unicenter.
    \item \textbf{Red pública de un café}. Por último, realizamos una captura
    en la red pública de Wi-Fi del café \emph{Starbucks} de Av. Callao y Perón.
\end{enumerate}



Las capturas fueron analizadas utilizando conceptos de Teoría de la
Información. Para esto, en cada caso, se modelaron a partir del tráfico
capturado las siguientes dos fuentes de información, considerando distintos
aspectos que se deseaba estudiar:

\begin{enumerate}
    \item \textbf{Fuente $\mathcal{S}$}. Consta de dos símbolos, $\lbrace
    s_{\text{BROADCAST}},\ s_{\text{UNICAST}} \rbrace$. Cada paquete 
    capturado en la red se considera un símbolo; aquellos con destino
    \emph{broadcast} (dirección MAC \texttt{ff:ff:ff:ff:ff:ff}) corresponden
    al símbolo $s_{\text{BROADCAST}}$, mientras que los demás corresponden
    al símbolo $s_{\text{UNICAST}}$.

    La probabilidad de la aparición de cada símbolo, y por consiguiente
    la entropía de la fuente, se calculó en base a la frecuencia relativa
    de cada uno de ellos.
    
    \item \textbf{Fuente $\mathcal{S}_1$}. Esta fuente se modeló teniendo en
    cuenta las direcciones IP de los paquetes ARP capturados. Cada dirección de IP es un símbolo en esta fuente.
    De todos los paquetes capturados, se retuvieron solo aquellos que cumplieran las siguientes condiciones:
    \begin{enumerate}
     \item fueran requests ARP (WHO-HAS?), ya que, si tomáramos las replies, tendríamos
     un sesgo muy fuerte para el equipo que está haciendo la captura si la red está switcheada
     o si es WiFi no abierto. Esto es porque las replies ARP son unicast, y solo llegan a los
     equipos target si no capturamos en un lugar ``privilegiado'' de la topología de red. Además,
     si pudiéramos capturar todas las replies IS-AT, no obtendríamos información nueva, ya que las
     replies se corresponden con las requests (salvo las ``gratuitous'' (ver siguiente item).
     \item no fueran ``gratuitous'' (donde target y source son la misma IP), ya que no 
     aportan información relevante sobre la topología de la red ni reflejan el nivel de actividad de los que los envían.
    \end{enumerate}

    Para calcular la probabilidad de la aparición de cada símbolo se usó la frecuencia relativa de cada uno.
    Se consideraron tres opciones: contar solo cuando una IP aparece como target, contar solo cuando aparece como source, o ambas. Decidimos 
    usar esta última opción, porque quien hace una request ARP puede ser cualquier nodo, sea distinguido o no. Pero lo que queremos
    ver es cuál (o cuáles) nodo ``habla'' más veces con otros nodos de la red.
    
\end{enumerate}

