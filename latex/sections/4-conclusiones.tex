% -*- root: ../informe.tex -*-

\section{Conclusiones}

% De haber diferentes tecnologías entre las redes capturadas, ¿Aprecia alguna
    % diferencia desde el punto de vista de las fuentes de información
    % analizadas?
    
 La fuente de información S cambia notablemente según el tipo de tecnología de la red donde se efectúa la captura. En el caso de redes switcheadas, se ve una proporción mucho mayor de paquetes broadcast que de unicast. En cambio, si la tecnología de la red es de WiFi abierto, se pueden capturar muchos más paquetes unicast. Lo mismo sucedería si la red cableada tuviera un hub en lugar de un switch.
 
 La fuente de información $S_1$, en cambio, no presenta diferencias según el tipo de tecnología, ya que aquí se analizan solo paquetes broadcast.
 
% De haber diferentes tamaños de redes, ¿Aprecia alguna diferencia desde el
    % punto de vista de las fuentes de información analizadas?
% ¿Que importancia tiene la entropía en la capacidad de detectar símbolos
    % (nodos/hosts) distinguidos?
% ¿Hay alguna relación entre la entropía y la cantidad de nodos (distinguidos
    % y no distinguidos)?

\begin{figure}[H]
    \figdef[dim]{figures/general_entropy_vs_nodes}
    % \caption{Cantidad de información por símbolo y entropía de $\mathcal{S}_1'$.}
    % \label{res:esc3:s1prime}
\end{figure}

