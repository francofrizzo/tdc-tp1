% -*- root: ../informe.tex -*-

\section{Conclusiones}

A continuación, presentamos algunas de las conclusiones generales acerca de
los resultados obtenidos y la metodología empleada, a las que arribamos luego
de comparar los análisis de cada una de las tres redes estudiadas.

% De haber diferentes tecnologías entre las redes capturadas, ¿Aprecia alguna
    % diferencia desde el punto de vista de las fuentes de información
    % analizadas?

Uno de los ejes de comparación interesante se origina a partir de las
diferentes tecnologías subyacentes a las redes analizadas. En este sentido,
podemos notar que la fuente de información $\mathcal{S}$ se ve notablemente
afectada por este factor. En el caso de la red \emph{switcheada}, se ve una
proporción mucho mayor de paquetes \emph{broadcast} que de \emph{unicast}, ya
que estos últimos son \emph{forwardeados} únicamente hacia su destino,
impidiendo ser capturados desde un \emph{host} arbitrario. En cambio, si la
tecnología de la red es de \emph{WiFi} abierto, se pueden capturar muchos más
paquetes \emph{unicast}. Lo mismo sucedería si la red cableada tuviera un
\emph{hub} en lugar de un \emph{switch}. La Figura \ref{general:s} muestra
una comparación entre la cantidad de información aportada, en cada red,
por los símbolos $s_{\mathtt{UNICAST}}$ y $s_{\mathtt{BROADCAST}}$.

\begin{figure}[H]
    \figdef[dim]{figures/general_s_fig}
    \caption{Cantidad de información por símbolo para la fuente $\mathcal{S}$
        de cada red.}
    \label{general:s}
\end{figure}

La fuente de información $\mathcal{S}_1$, en cambio, no presenta diferencias
evidentes según el tipo de tecnología, ya que aquí se analizan solo paquetes
\emph{broadcast}, que en ambos casos son capturados por todos los
\emph{hosts}. Los diferentes resultados obtenidos para esta fuente son mejor
explicados por las diferentes características topológicas y el tipo de
actividad de las redes.

% De haber diferentes tamaños de redes, ¿Aprecia alguna diferencia desde el
    % punto de vista de las fuentes de información analizadas?
% ¿Hay alguna relación entre la entropía y la cantidad de nodos (distinguidos
    % y no distinguidos)?

Otro aspecto interesante a comparar tiene que ver con los distintos tamaños de
redes analizadas. En cuanto a la fuente $\mathcal{S}$, como posee una cantidad
fija de símbolos (dos), su entropía máxima es siempre 1, independientemente de
la cantidad de \emph{hosts} de la red. Además, si bien los resultados obtenidos
para esta fuente en las tres capturas son claramente distintos, las
características disímiles de las redes hacen que sea difícil analizar esto en
términos de la cantidad de nodos cada una; creemos este hecho se explica mejor
a partir de otras características, por ejemplo, las ya mencionadas diferencias
en las tecnologías subyacentes.

En cuanto a la fuente $\mathcal{S}_1$, sí se observa una clara relación
con la cantidad de nodos de cada red; como a cada nodo le corresponde un
símbolo en esta fuente, una mayor cantidad de nodos incrementa
la cantidad de información que aporta cada uno, y por consiguiente, también la
entropía de la fuente. En la Figura \ref{general:entropy_vs_nodes} puede
notarse esta relación.

\begin{figure}[H]
    \figdef[dim]{figures/general_entropy_vs_nodes}
    \caption{Entropía de $\mathcal{S}_1$ según la cantidad de nodos.}
    \label{general:entropy_vs_nodes}
\end{figure}

En términos teóricos, podría esperarse también que la cantidad de nodos
distinguidos por sobre el total afecte a la entropía de la fuente; a más nodos
destacados, la fuente se aleja más de una fuente equiprobable y se vuelve
menos predecible, por lo que su entropía debería aumentar. No obstante, es
difícil corroborar esta hipótesis con los experimentos llevados a cabo, ya que
hay en juego demasiados factores que afectan a la entropía de una manera más
drástica.

% ¿Que importancia tiene la entropía en la capacidad de detectar símbolos
    % (nodos/hosts) distinguidos?

El método desarrollado para la detección de nodos distinguidos, basado en la
entropía de la fuente $\mathcal{S}_1$, dio resultados dispares. En el caso de
la red del \emph{shopping}, que posee una cantidad bastante grande de nodos
pero una topología sencilla, el resultado obtenido fue muy preciso y permitió
detectar al \emph{default gateway} con gran claridad. En el escenario de la
red cableada, que tiene una topología mucho más compleja, se detectaron muchos
nodos distinguidos; si bien algunos tienen claramente una función destacada
(aunque no necesariamente la de \emph{default gateways}), en otros este no
parece ser el caso, por lo que se hace necesario complementar el método con
información adicional sobre la estructura de la red. Por último, en el caso de
la captura realizada en el café, el resultado fue muy inexacto, ya que se vio
fuertemente afectado por la presencia de paquetes de más de una red y por
comportamientos anómalos de alguos de los \emph{hosts}. Claramente, el método
aporta información útil, pero no es una solución precisa en el caso general.
