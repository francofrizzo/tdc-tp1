\documentclass[%
    final,
    notitlepage,
    narroweqnarray,
    inline,
    twoside,
]{ieee}
\usepackage{ieeefig}
\usepackage{pgfplots}
\pgfplotsset{compat=newest}
\usepackage[utf8]{inputenc}
\usepackage[spanish]{babel}
\usepackage{amsmath}
\usepackage{float}
\usepackage{siunitx} % http://mirrors.ctan.org/install/macros/latex/contrib/siunitx.tds.zip
\sisetup{
    binary-units    = true,
    round-mode      = places,
    round-precision = 2,
    output-decimal-marker = {,}
}

% Style to select only points from #1 to #2 (inclusive)
\pgfplotsset{select coords between index/.style 2 args={
    x filter/.code={
        \ifnum\coordindex<#1\def\pgfmathresult{}\fi
        \ifnum\coordindex>#2\def\pgfmathresult{}\fi
    }
}}

\def\keywordsname{Palabras clave}

\begin{document}

\title[Trabajo Práctico 1: Wiretapping]{%
       Trabajo Práctico 1: \emph{Wiretapping}}

\author[FRIZZO, MONTEPAGANO, PONDAL]{Franco Frizzo, Pablo Montepagano
\and{}e Iván Pondal}

\journal{Teoría de las Comunicaciones (DC - FCEyN - UBA)}
\titletext{, Trabajo Práctico 1: \emph{Wiretapping}}

\maketitle

\begin{abstract}
En el presente trabajo, realizamos un análisis del tráfico de redes de
computadoras, modelando las mismas como fuentes de información con las
herramientas teóricas brindadas por la teoría de la información, para así
extraer diversas conclusiones acerca de sus características.
En particular, se estudió el comportamiento del protocolo ARP, la
posibilidad de distinguir a partir de este último aquellos nodos que
tuvieran un papel destacado dentro de la red, y la interpretación
en este contexto de conceptos como cantidad de información y entropía de una
fuente.
\end{abstract}

\begin{keywords}
Wiretapping, redes, teoría de la información, entropía, ARP.
\end{keywords}

% \begin{figure}[htb]
% \figdef{figura}
% \caption{Figura de ejemplo.}
% \label{fig-example}
% % \end{figure}Z

% -*- root: ../informe.tex -*-

\section{Introducción}

En el presente trabajo, nos proponemos conocer mejor la infraestructura
subyacente a las redes de computadoras a partir de capturas de tráfico y su
posterior análisis. Haremos el análisis de tres redes de diferentes tamaños,
características y tecnologías, para así poder contrastar los resultados
obtenidos.

El elemento en común entre todas las redes estudiadas fue la utilización del
protocolo de red \texttt{IP}, lo cual nos permitió tener en cuenta,
especialmente, el comportamiento del protocolo de control \texttt{ARP}
(\textit{Address Resolution Protocol}), para analizar la naturaleza de las
comunicaciones entre los nodos y extraer conclusiones acerca de la topología
de la red. Especialmente, se estudió la posibilidad de detectar, a partir de
este protocolo, aquellos nodos que jugaran un papel destacado dentro de la
red.

Una herramienta clave para el análisis de los datos fue el empleo de conceptos
de Teoría de la Información, como el de entropía, para evaluar la importancia
relativa de los distintos nodos de la red. Se presentan los resultados así
obtenidos para cada una de las redes, como así también un análisis comparativo
y una interpretación general de los mismos.

\section{Metodología}

Para realizar los experimentos, utilizamos dos herramientas: \emph{Wireshark},
una aplicación de código abierto que permite capturar y analizar los paquetes
que transitan por una red, y \emph{Scapy}, una herramienta escrita en
Python para manipular y trabajar con los paquetes capturados.

Se capturó todo el tráfico de tres redes redes diferentes, en los tres
escenarios que se detallan a continuación.

\begin{enumerate}
    \item \textbf{Red cableada}. Realizamos esta captura en la red Ethernet
    de los laboratorios del Departamento de Computación (FCEyN, UBA). Se trata
    de una red grande y con bastante tráfico.
    \item \textbf{Red pública de un shopping}. Capturamos también el tráfico
    de la red Wi-Fi pública del \emph{shopping} Unicenter.
    \item \textbf{Red pública de un café}. Por último, realizamos una captura
    en la red pública de Wi-Fi del café \emph{Starbucks} de Av. Callao y Perón.
\end{enumerate}

En los tres casos, se filtraron los resultados para considerar solo los
paquetes relativos al protocolo ARP (\emph{Address Resolution Protocol}).
(Esto capaz queda explicado antes).

Los resultados fueron analizados utilizando conceptos de Teoría de la
Información. Para esto, en cada caso, se modelaron a partir del tráfico
capturado las siguientes dos fuentes de información, considerando distintos
aspectos que se deseaba estudiar:

\begin{enumerate}
    \item \textbf{Fuente $\mathcal{S}$}. Consta de dos símbolos, $\lbrace
    s_{\text{BROADCAST}},\ s_{\text{UNICAST}} \rbrace$. Cada paquete ARP
    capturado en la red se considera un símbolo; aquellos con destino
    \emph{broadcast} (dirección MAC \texttt{ff:ff:ff:ff:ff:ff}) corresponden
    al símbolo $s_{\text{BROADCAST}}$, mientras que los demás corresponden
    al símbolo $s_{\text{UNICAST}}$.

    La probabilidad de la aparición de cada símbolo, y por consiguiente
    la entropía de la fuente, se calculó en base a la frecuencia relativa
    de cada uno de ellos.
    \item \textbf{Fuente $\mathcal{S}_1$}. Esta fuente se modeló teniendo en
    cuenta las direcciones IP de los paquetes ARP capturados.
\end{enumerate}


\section{Resultados}

\newread\tmp

\openin\tmp=../exp/shopping.s.entropy
\read\tmp to \ShoppingSEntropy
\closein\tmp

\openin\tmp=../exp/shopping.s1.entropy
\read\tmp to \ShoppingSOneEntropy
\closein\tmp

\openin\tmp=../exp/starbucks.s.entropy
\read\tmp to \StarbucksSEntropy
\closein\tmp

\openin\tmp=../exp/starbucks.s1.entropy
\read\tmp to \StarbucksSOneEntropy
\closein\tmp

\openin\tmp=../exp/wired_lan.s.entropy
\read\tmp to \WiredLanSEntropy
\closein\tmp

\openin\tmp=../exp/wired_lan.s1.entropy
\read\tmp to \WiredLanSOneEntropy
\closein\tmp

\subsection{Escenario 1 (Red cableada)}

\subsubsection{Fuente S}

En esta red, se capturaron muchos más paquetes broadcast que unicast. Se puede ver en la figura \ref{res:esc1:fig1}
que los paquetes unicast aportan más información que los broadcast. Esto se explica porque se trata de una red switcheada. Por lo tanto, los paquetes unicast que tienen como destino otros hosts, no llegan al equipo de captura. Sin embargo, como en la red hay una cantidad elevada de hosts, se percibe una gran cantidad de broadcast. Como el equipo de captura no se usó para generar una cantidad relevante de tráfico, se capturó más tráfico broadcast de la red, que el tráfico unicast proveniente o destinado a nuestro host de captura.

Como es posible observar en la Figura \ref{res:esc1:fig1}, la entropía de la fuente S no es máxima.

\begin{figure}[h]
	\figdef[dim]{figures/wired_lan_s_fig}
	\caption{Cantidad de información por símbolo y entropia de $\mathcal{S}$.}
    \label{res:esc1:fig1}
\end{figure}
    
    


\begin{tikzpicture}[baseline]
    \begin{axis}[
            title={},
            xlabel={Símbolo (Dirección IP)},
            ylabel={Cantidad de información},
            scaled x ticks=false,
            scaled y ticks=false,
            width=0.45\textwidth,
            height=0.35\textwidth,
            legend pos=outer north east,
            legend cell align=left,
            ymin=0,
            xtick=data,
            xticklabels from table={../exp/wired_lan.s1.information}{IP},
            x tick label style={rotate=80,anchor=east,font=\small}
        ]
        \addplot [
                ybar,
                fill=blue!10,
                draw=blue,
                select coords between index={0}{20}
            ] table[x=X-Pos,y=Information]{../exp/wired_lan.s1.information};

        \coordinate (A) at (axis cs:0,\WiredLanSOneEntropy);
        \coordinate (O1) at (rel axis cs:0,0);
        \coordinate (O2) at (rel axis cs:1,0);

        \draw [blue, thick] (A -| O1) -- (A -| O2);

    \end{axis}
\end{tikzpicture}

\begin{figure}[H]
    \figdef[dim]{figures/wired_lan_arp_fig}
    \caption{Red de mensajes \texttt{ARP} para captura de red cableada.}
\end{figure}




\subsection{Escenario 2 (\emph{Shopping})}

La principal motivación para estudiar esta red fue poder observar un medio con
mucha actividad, el hecho de que se tratara de una red inalámbrica permitió además
contrastar las diferencias frente a la primer captura sobre una red cableada.

Primero se analizarán los resultados obtenidos para la fuente $\mathcal{S}$.

\begin{figure}[h]
	\figdef[dim]{figures/shopping_s_fig}
	\caption{Cantidad de información por símbolo y entropía de $\mathcal{S}$.}
    \label{res:esc2:fig1}
\end{figure}


Como es posible observar en la Figura \ref{res:esc2:fig1} la entropía de
$\mathcal{S}$ no es máxima. La misma posee un valor relativamente inferior lo
cual indica que la fuente es bastante predecible. Se puede apreciar cómo la
cantidad de información que aporta $s_{\texttt{UNICAST}}$ es mínima con respecto
a $s_{\texttt{BROADCAST}}$. Esto se debe al bajo número de paquetes con destino
broadcast en la red.

Una posible hipótesis a este resultado es el hecho de que se trate de la red
inalámbrica de un centro comercial. Es razonable asumir que los hosts conectados
a la misma más que comunicarse entre ellos estarán accediendo mediante el
gateway a Internet.

Para ello, los mismos necesitan la dirección física asociada a la dirección
\texttt{IP} del gateway. Aquí es donde entra en juego \texttt{ARP}, donde los
hosts a través de mensajes broadcast \texttt{WHO-HAS} consultan por ella. Cuando
este mensaje llega al default gateway además de responder con su dirección
física a cada host que consultó por él, el dispositivo guarda la asociación
entre la dirección \texttt{IP} que originó el mensaje junto a su dirección física.

De esta manera, en un principio los únicos mensajes broadcast presentes son los
que resultan de los paquetes \texttt{ARP} generados por los hosts con el fin de
tener la dirección física del gateway. Una vez hecho esto, el gateway ya conoce
a los hosts por lo tanto cualquier comunicación entre ambos resulta en un
intercambio de paquetes unicast.

Habiendo dicho esto, asumiendo que la hipótesis es cierta, sería correcto
afirmar que existe una relación entre los protocolos de control como
\texttt{ARP} y la fuente $\mathcal{S}$. Si hubiera habido comunicación entre los
nodos de la red entonces el número de broadcasts habría sido necesariamente
mayor dado que en lugar de sólo consultar por la dirección física del gateway
también se lo habría hecho por el resto de los hosts.

A continuación se analizarán los resultados obtenidos para la fuente
$\mathcal{S}_1$.

\begin{figure}[h]
	\figdef[dim]{figures/shopping_s1_fig}
	\caption{Cantidad de información por símbolo y entropía de $\mathcal{S}_1$.}
    \label{res:esc2:fig2}
\end{figure}

Estudiando el tráfico \texttt{ARP} de la captura de esta red se notaron algunas
características que merecen ser mencionadas. La primera es que corroborando la
hipótesis sugerida en el análisis anterior, los mensajes del tipo
\texttt{WHO-HAS} iban todos dirigidos al gateway de la red. La segunda,
fuertemente relacionada con la primera, es el hecho de que el gateway es el
único generando respuestas \texttt{IS-AT} a los hosts y prácticamente no realiza
ningún pedido \texttt{WHO-HAS}.

Ambos comportamientos pueden ser justificados con la hipótesis sobre el tipo de
comunicaciones que se establecen en una red inalámbrica dentro de un centro
comercial. Además, buscando información sobre los routers inalámbricos
utilizados para este tipo de instalaciones se encontró que el \emph{timeout} de
las tablas \texttt{ARP} puede llegar a valores de hasta 4 horas. Esto tiene
sentido puesto que estos dispositivos tienen la capacidad de almacenamiento
necesaria para no tener que preocuparse por quedarse sin espacio para las
entradas.

El modelo utilizado para la fuente $\mathcal{S}_1$ dio un total de 30 nodos en la
red. En la Figura \ref{res:esc2:fig2} se muestran los primeros 20 ordenados por
la cantidad de información que brindan, donde se puede apreciar como hay un
único dispositivo cuyo nivel de información está por debajo del de la entropía
calculada.

Este nodo distinguido resulta ser el gateway de la red estudiada. Su bajo aporte
de información tiene sentido puesto que con el modelo utilizado por cada mensaje
\texttt{WHO-HAS} que lo tenga como destino u origen aumenta la probabilidad de
ocurrencia de su símbolo. Por todo lo discutido anteriormente esto
necesariamente lo vuelve el símbolo con mayor probabilidad de aparición y como
consecuencia el que menor información aporta.

Por último, en la Figura \ref{res:esc2:fig3} se tiene la visualización del
tráfico \texttt{ARP} donde el tamaño de los nodos es proporcional a la
probabilidad de sus símbolos en la fuente $\mathcal{S}_1$. Aquí queda más que
evidente el rol de gateway del nodo distinguido, donde todo el tráfico del
protocolo es entre los hosts y el mismo.

Con respecto a la posibilidad de utilizar esta fuente como método para encontrar
los default gateways, como en este caso resultó efectivo uno podría verse
tentado a decir que el mismo es efectivo en la tarea. Sin embargo, existen muchos
factores que podrían llegar a alterar los resultados obtenidos generando
conclusiones incongruentes. Por ejemplo, la captura sobre la cual se realizó
todo este estudio fue hecha en un punto donde llegaba la señal de un único
\emph{access point}. Si hubieran habido más de estos, dependiendo dónde se
realizaba físicamente la captura se habrían obtenido más o menos paquetes
dirigidos a los mismos. Por lo tanto podría haber ocurrido que habiendo dos
default gateways por el simple hecho de que no llegaran a capturarse suficientes
paquetes dirigidos a uno de ellos el símbolo representando al mismo tuviera una
probabilidad asociada muy baja generando entonces mucha información en la fuente
$\mathcal{S}_1$ quedando por encima de la entropía calculada.

\begin{figure}[h!]
	\figdef[dim]{figures/shopping_arp_fig}
    \caption{Red de mensajes \texttt{ARP} para captura del shopping.}
    \label{res:esc2:fig3}
\end{figure}

\vfill % used to avoid funny stretching between paragraphs

\subsection{Escenario 3 (\emph{Starbucks})}

\begin{tikzpicture}[baseline]
    \begin{axis}[
            title={},
            xlabel={Símbolo (Destino)},
            ylabel={Cantidad de información},
            scaled x ticks=false,
            scaled y ticks=false,
            width=0.45\textwidth,
            height=0.35\textwidth,
            bar width=2cm,
            ymin=0,
            enlarge x limits=0.65,
            xtick=data,
            major tick length=0,
            xticklabels={$s_{\texttt{UNICAST}}$, $s_{\texttt{BROADCAST}}$},
            x tick label style={rotate=80,anchor=east,font=\small},
            legend entries={$H(\mathcal{S})$,$H_{\max}(\mathcal{S})$},
            legend style={
                legend cell align=left,
                legend pos=north west
            }
        ]
        \addplot[
                ybar,
                fill=blue!10,
                draw=blue,
                forget plot
            ] table[x=IsBroadcast,y=Information]{../exp/starbucks.s.information};
        \addplot[mark=none, blue, thick, update limits=false] {\StarbucksSEntropy};
        \addplot[mark=none, blue, thick, dashed, update limits=false] {1};
    \end{axis}
\end{tikzpicture}

\begin{tikzpicture}[baseline]
    \begin{axis}[
            title={},
            xlabel={Símbolo (Dirección IP)},
            ylabel={Cantidad de información},
            scaled x ticks=false,
            scaled y ticks=false,
            width=0.45\textwidth,
            height=0.35\textwidth,
            legend pos=outer north east,
            legend cell align=left,
            ymin=0,
            xtick=data,
            xticklabels from table={../exp/starbucks.s1.information}{IP},
            x tick label style={rotate=80,anchor=east,font=\small}
        ]
        \addplot [ybar, fill=blue!10, draw=blue] table[x=X-Pos,y=Information]
                {../exp/starbucks.s1.information};

        \coordinate (A) at (axis cs:0,\StarbucksSOneEntropy);
        \coordinate (O1) at (rel axis cs:0,0);
        \coordinate (O2) at (rel axis cs:1,0);

        \draw [blue, thick] (A -| O1) -- (A -| O2);

    \end{axis}
\end{tikzpicture}

\begin{figure}[H]
    \figdef[dim]{figures/starbucks_arp_fig}
    \caption{Red de mensajes \texttt{ARP} para captura del Starbucks.}
\end{figure}

% -*- root: ../informe.tex -*-

\section{Conclusiones}

% De haber diferentes tecnologías entre las redes capturadas, ¿Aprecia alguna
    % diferencia desde el punto de vista de las fuentes de información
    % analizadas?
    
 La fuente de información S cambia notablemente según el tipo de tecnología de la red donde se efectúa la captura. En el caso de redes switcheadas, se ve una proporción mucho mayor de paquetes broadcast que de unicast. En cambio, si la tecnología de la red es de WiFi abierto, se pueden capturar muchos más paquetes unicast. Lo mismo sucedería si la red cableada tuviera un hub en lugar de un switch.
 
 La fuente de información $S_1$, en cambio, no presenta diferencias según el tipo de tecnología, ya que aquí se analizan solo paquetes broadcast.
 
% De haber diferentes tamaños de redes, ¿Aprecia alguna diferencia desde el
    % punto de vista de las fuentes de información analizadas?
% ¿Que importancia tiene la entropía en la capacidad de detectar símbolos
    % (nodos/hosts) distinguidos?
% ¿Hay alguna relación entre la entropía y la cantidad de nodos (distinguidos
    % y no distinguidos)?

\begin{figure}[H]
    \figdef[dim]{figures/general_entropy_vs_nodes}
    % \caption{Cantidad de información por símbolo y entropía de $\mathcal{S}_1'$.}
    % \label{res:esc3:s1prime}
\end{figure}



\bibliographystyle{IEEEbib}
\bibliography{informe}

\end{document}
