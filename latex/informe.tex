\documentclass[%
    final,
    notitlepage,
    narroweqnarray,
    inline,
    twoside,
]{ieee}
\usepackage{ieeefig}
\usepackage{pgfplots}
\pgfplotsset{compat=newest}
\usepackage[utf8]{inputenc}
\usepackage[spanish]{babel}
\usepackage{amsmath}
\usepackage{float}
\usepackage{siunitx} % http://mirrors.ctan.org/install/macros/latex/contrib/siunitx.tds.zip
\sisetup{
    binary-units    = true,
    round-mode      = places,
    round-precision = 2,
    output-decimal-marker = {,}
}

% Style to select only points from #1 to #2 (inclusive)
\pgfplotsset{select coords between index/.style 2 args={
    x filter/.code={
        \ifnum\coordindex<#1\def\pgfmathresult{}\fi
        \ifnum\coordindex>#2\def\pgfmathresult{}\fi
    }
}}

\def\keywordsname{Palabras clave}

\begin{document}

\title[Trabajo Práctico 1: Wiretapping]{%
       Trabajo Práctico 1: \emph{Wiretapping}}

\author[FRIZZO, MONTEPAGANO, PONDAL]{Franco Frizzo, Pablo Montepagano
\and{}e Iván Pondal}

\journal{Teoría de las Comunicaciones (DC - FCEyN - UBA)}
\titletext{, Trabajo Práctico 1: \emph{Wiretapping}}

\maketitle

\begin{abstract}
En el presente trabajo, realizamos un análisis del tráfico de redes de
computadoras, modelando las mismas como fuentes de información con las
herramientas teóricas brindadas por la teoría de la información, para así
extraer diversas conclusiones acerca de sus características, en particular,
sobre el comportamiento del protocolo ARP, la posibilidad de distinguir los
roles de ciertos nodos de la red a partir de este último, y la interpretación
en este contexto de conceptos como cantidad de información y entropía de una
fuente.

\end{abstract}

\begin{keywords}
Wiretapping, redes, teoría de la información, entropía, ARP.
\end{keywords}

% \begin{figure}[htb]
% \figdef{figura}
% \caption{Figura de ejemplo.}
% \label{fig-example}
% % \end{figure}Z

% -*- root: ../informe.tex -*-

\section{Introducción}

En el presente trabajo, nos proponemos conocer mejor la infraestructura
subyacente a las redes de computadorasa partir de capturas de tráfico y su posterior análisis.
Haremos el análisis de tres
redes de diferentes tamaños, características y tecnologías, para así poder
contrastar los resultados obtenidos.

El elemento en común entre todas las redes estudiadas fue la utilización del
protocolo de red \texttt{IP}, lo cual nos permitió tener en cuenta,
especialmente, el comportamiento del protocolo de control \texttt{ARP}
(\textit{Address Resolution Protocol}), para analizar la naturaleza de las
comunicaciones entre los nodos y extraer conclusiones acerca de la topología
de la red. Especialmente, se estudió la posibilidad de detectar, a partir de
este protocolo, aquellos nodos que jugaran un papel destacado dentro de la
red.
g
Una herramienta clave para el análisis de los datos fue el empleo de conceptos
de Teoría de la Información, como el de entropía, para evaluar la importancia
relativa de los distintos nodos de la red. Se presentan los resultados así
obtenidos para cada una de las redes, como así también un análisis comparativo
y una interpretación general de los mismos.

\section{Metodología}

Para realizar los experimentos, utilizamos dos herramientas: \emph{Wireshark},
una aplicación de código abierto que permite capturar y analizar los paquetes
que transitan por una red, y \emph{Scapy}, una herramienta escrita en
Python para manipular y trabajar con los paquetes capturados.

Se capturó todo el tráfico de tres redes redes diferentes, en los tres
escenarios que se detallan a continuación.

\begin{enumerate}
    \item \textbf{Red cableada}. Realizamos esta captura en la red Ethernet
    de los laboratorios del Departamento de Computación (FCEyN, UBA). Se trata
    de una red grande y con bastante tráfico.
    \item \textbf{Red pública de un shopping}. Capturamos también el tráfico
    de la red Wi-Fi pública del \emph{shopping} Unicenter.
    \item \textbf{Red pública de un café}. Por último, realizamos una captura
    en la red pública de Wi-Fi del café \emph{Starbucks} de Av. Callao y Perón.
\end{enumerate}



Las capturas fueron analizadas utilizando conceptos de Teoría de la
Información. Para esto, en cada caso, se modelaron a partir del tráfico
capturado las siguientes dos fuentes de información, considerando distintos
aspectos que se deseaba estudiar:

\begin{enumerate}
    \item \textbf{Fuente $\mathcal{S}$}. Consta de dos símbolos, $\lbrace
    s_{\text{BROADCAST}},\ s_{\text{UNICAST}} \rbrace$. Cada paquete
    capturado en la red se considera un símbolo; aquellos con destino
    \emph{broadcast} (dirección MAC \texttt{ff:ff:ff:ff:ff:ff}) corresponden
    al símbolo $s_{\text{BROADCAST}}$, mientras que los demás corresponden
    al símbolo $s_{\text{UNICAST}}$.

    La probabilidad de la aparición de cada símbolo, y por consiguiente
    la entropía de la fuente, se calculó en base a la frecuencia relativa
    de cada uno de ellos.

    \item \textbf{Fuente $\mathcal{S}_1$}. Esta fuente se modeló teniendo en
    cuenta las direcciones IP de los paquetes ARP capturados. Cada dirección
    de IP es un símbolo en esta fuente. De todos los paquetes capturados, se
    retuvieron solo aquellos que cumplieran las siguientes condiciones:
    \begin{enumerate}
     \item fueran requests ARP (WHO-HAS?), ya que, si tomáramos las replies, tendríamos
     un sesgo muy fuerte para el equipo que está haciendo la captura si la red está switcheada
     o si es WiFi no abierto. Esto es porque las replies ARP son unicast, y solo llegan a los
     equipos target si no capturamos en un lugar ``privilegiado'' de la topología de red. Además,
     si pudiéramos capturar todas las replies IS-AT, no obtendríamos información nueva, ya que las
     replies se corresponden con las requests (salvo las ``gratuitous'' (ver siguiente item).
     \item no fueran ``gratuitous'' (donde target y source son la misma IP), ya que no
     aportan información relevante sobre la topología de la red ni reflejan el nivel de actividad de los que los envían.
    \end{enumerate}

    Para calcular la probabilidad de la aparición de cada símbolo se usó la frecuencia relativa de cada uno.
    Se consideraron tres opciones: contar solo cuando una IP aparece como target, contar solo cuando aparece como source, o ambas. Decidimos
    usar esta última opción, porque quien hace una request ARP puede ser cualquier nodo, sea distinguido o no. Pero lo que queremos
    ver es cuál (o cuáles) nodo ``habla'' más veces con otros nodos de la red.

\end{enumerate}


\section{Resultados}


\newread\tmp
\openin\tmp=../exp/shopping_0.entropy
\read\tmp to \ShoppingAEntropy
\closein\tmp

\openin\tmp=../exp/shopping_1.entropy
\read\tmp to \ShoppingBEntropy
\closein\tmp

\openin\tmp=../exp/starbucks_0.entropy
\read\tmp to \StarbucksEntropy
\closein\tmp

\openin\tmp=../exp/wired_lan_0.entropy
\read\tmp to \WiredLanEntropy
\closein\tmp

\subsection{Escenario 1 (Red cableada)}

\begin{tikzpicture}[baseline]
    \begin{axis}[
            title={},
            xlabel={Símbolo (Dirección IP)},
            ylabel={Cantidad de información},
            scaled x ticks=false,
            scaled y ticks=false,
            width=0.45\textwidth,
            height=0.35\textwidth,
            legend pos=outer north east,
            legend cell align=left,
            ymin=0,
            xtick=data,
            xticklabels from table={../exp/wired_lan_0.information}{IP},
            x tick label style={rotate=80,anchor=east,font=\small}
        ]
        \addplot [ybar, fill=blue!10, draw=blue] table[x=X-Pos,y=Information]
                {../exp/wired_lan_0.information};

        \coordinate (A) at (axis cs:0,\WiredLanEntropy);
        \coordinate (O1) at (rel axis cs:0,0);
        \coordinate (O2) at (rel axis cs:1,0);

        \draw [blue, thick] (A -| O1) -- (A -| O2);

    \end{axis}
\end{tikzpicture}

\begin{figure}[H]
	\includegraphics[scale=0.4]{figures/wired_lan.pdf}
	\caption{Red de mensajes \texttt{ARP} para captura de red cableada.}
\end{figure}

\subsection{Escenario 2 (\emph{Shopping})}

\begin{tikzpicture}[baseline]
    \begin{axis}[
            title={},
            xlabel={Símbolo (Dirección IP)},
            ylabel={Cantidad de información},
            scaled x ticks=false,
            scaled y ticks=false,
            width=0.45\textwidth,
            height=0.35\textwidth,
            legend pos=outer north east,
            legend cell align=left,
            ymin=0,
            xtick=data,
            xticklabels from table={../exp/shopping_0.information}{IP},
            x tick label style={rotate=80,anchor=east,font=\small}
        ]
        \addplot [ybar, fill=blue!10, draw=blue] table[x=X-Pos,y=Information]
                {../exp/shopping_0.information};

        \coordinate (A) at (axis cs:0,\ShoppingAEntropy);
        \coordinate (O1) at (rel axis cs:0,0);
        \coordinate (O2) at (rel axis cs:1,0);

        \draw [blue, thick] (A -| O1) -- (A -| O2);

    \end{axis}
\end{tikzpicture}

\begin{figure}[H]
	\includegraphics[scale=0.4]{figures/shopping.pdf}
	\caption{Red de mensajes \texttt{ARP} para captura del shopping.}
\end{figure}

\subsection{Escenario 3 (\emph{Starbucks})}

\begin{tikzpicture}[baseline]
    \begin{axis}[
            title={},
            xlabel={Símbolo (Dirección IP)},
            ylabel={Cantidad de información},
            scaled x ticks=false,
            scaled y ticks=false,
            width=0.45\textwidth,
            height=0.35\textwidth,
            legend pos=outer north east,
            legend cell align=left,
            ymin=0,
            xtick=data,
            xticklabels from table={../exp/starbucks_0.information}{IP},
            x tick label style={rotate=80,anchor=east,font=\small}
        ]
        \addplot [ybar, fill=blue!10, draw=blue] table[x=X-Pos,y=Information]
                {../exp/starbucks_0.information};

        \coordinate (A) at (axis cs:0,\StarbucksEntropy);
        \coordinate (O1) at (rel axis cs:0,0);
        \coordinate (O2) at (rel axis cs:1,0);

        \draw [blue, thick] (A -| O1) -- (A -| O2);

    \end{axis}
\end{tikzpicture}

\begin{figure}[H]
	\includegraphics[scale=0.4]{figures/starbucks.pdf}
	\caption{Red de mensajes \texttt{ARP} para captura del Starbucks.}
\end{figure}


\bibliographystyle{IEEEbib}
\bibliography{informe}

\end{document}
