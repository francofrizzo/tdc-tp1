\documentclass[%
    final,
    notitlepage,
    narroweqnarray,
    inline,
    twoside,
]{ieee}
\usepackage[utf8]{inputenc}
\usepackage[spanish]{babel}
\usepackage{ieeefig}
\usepackage{amsmath}

\def\keywordsname{Palabras clave}

\begin{document}

\title[Trabajo Práctico 1: Wiretapping]{%
       Trabajo Práctico 1: \emph{Wiretapping}}

\author[FRIZZO, MONTEPAGANO, PONDAL]{Franco Frizzo, Pablo Montepagano
\and{}e Iván Pondal}

\journal{Teoría de las Comunicaciones (DC - FCEyN - UBA)}
\titletext{, Trabajo Práctico 1: \emph{Wiretapping}}

\maketitle

\begin{abstract}
[Inserte resumen aquí]
\end{abstract}

\begin{keywords}
[palabra clave 1, palabra clave 2, palabra clave 3]
\end{keywords}

% -*- root: ../informe.tex -*-

\section{Introducción}

En el presente trabajo, nos proponemos conocer mejor la infraestructura
subyacente a las redes de computadoras a partir de capturas de tráfico y su
posterior análisis. Haremos el análisis de tres redes de diferentes tamaños,
características y tecnologías, para así poder contrastar los resultados
obtenidos.

El elemento en común entre todas las redes estudiadas fue la utilización del
protocolo de red \texttt{IP}, lo cual nos permitió tener en cuenta,
especialmente, el comportamiento del protocolo de control \texttt{ARP}
(\textit{Address Resolution Protocol}), para analizar la naturaleza de las
comunicaciones entre los nodos y extraer conclusiones acerca de la topología
de la red. Especialmente, se estudió la posibilidad de detectar, a partir de
este protocolo, aquellos nodos que jugaran un papel destacado dentro de la
red.

Una herramienta clave para el análisis de los datos fue el empleo de conceptos
de Teoría de la Información, como el de entropía, para evaluar la importancia
relativa de los distintos nodos de la red. Se presentan los resultados así
obtenidos para cada una de las redes, como así también un análisis comparativo
y una interpretación general de los mismos.

\section{Metodología}

Para realizar los experimentos, utilizamos dos herramientas: \emph{Wireshark},
una aplicación de código abierto que permite capturar y analizar los paquetes
que transitan por una red, y \emph{Scapy}, una herramienta escrita en
Python para manipular y trabajar con los paquetes capturados.

Se capturó todo el tráfico de tres redes redes diferentes, en los tres
escenarios que se detallan a continuación.

\begin{enumerate}
    \item \textbf{Red cableada}. Realizamos esta captura en la red Ethernet
    de los laboratorios del Departamento de Computación (FCEyN, UBA). Se trata
    de una red grande y con bastante tráfico.
    \item \textbf{Red pública de un shopping}. Capturamos también el tráfico
    de la red Wi-Fi pública del \emph{shopping} Unicenter.
    \item \textbf{Red pública de un café}. Por último, realizamos una captura
    en la red pública de Wi-Fi del café \emph{Starbucks} de Av. Callao y Perón.
\end{enumerate}

En los tres casos, se filtraron los resultados para considerar solo los
paquetes relativos al protocolo ARP (\emph{Address Resolution Protocol}).
(Esto capaz queda explicado antes).

Los resultados fueron analizados utilizando conceptos de Teoría de la
Información. Para esto, en cada caso, se modelaron a partir del tráfico
capturado las siguientes dos fuentes de información, considerando distintos
aspectos que se deseaba estudiar:

\begin{enumerate}
    \item \textbf{Fuente $\mathcal{S}$}. Consta de dos símbolos, $\lbrace
    s_{\text{BROADCAST}},\ s_{\text{UNICAST}} \rbrace$. Cada paquete ARP
    capturado en la red se considera un símbolo; aquellos con destino
    \emph{broadcast} (dirección MAC \texttt{ff:ff:ff:ff:ff:ff}) corresponden
    al símbolo $s_{\text{BROADCAST}}$, mientras que los demás corresponden
    al símbolo $s_{\text{UNICAST}}$.

    La probabilidad de la aparición de cada símbolo, y por consiguiente
    la entropía de la fuente, se calculó en base a la frecuencia relativa
    de cada uno de ellos.
    \item \textbf{Fuente $\mathcal{S}_1$}. Esta fuente se modeló teniendo en
    cuenta las direcciones IP de los paquetes ARP capturados.
\end{enumerate}



% \begin{figure}[htb]
% \figdef{figura}
% \caption{Figura de ejemplo.}
% \label{fig-example}
% \end{figure}

\bibliographystyle{IEEEbib}
\bibliography{informe}

\end{document}
